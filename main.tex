\documentclass{report}

\usepackage{amsmath, amssymb}
\usepackage[]{graphicx}
\usepackage[]{color}

%% commands from jss.cls
\newcommand\code{\bgroup\@makeother\_\@makeother\~\@makeother\$\@codex}
\def\@codex#1{{\normalfont\ttfamily\hyphenchar\font=-1 #1}\egroup}
% \let\code=\texttt
\let\proglang=\textsf
\newcommand{\pkg}[1]{{\fontseries{b}\selectfont #1}}
\newcommand{\email}[1]{\href{mailto:#1}{\normalfont\texttt{#1}}}
\newcommand{\E}{\mathsf{E}}
\newcommand{\VAR}{\mathsf{VAR}}
\newcommand{\COV}{\mathsf{COV}}
\newcommand{\Prob}{\mathsf{P}}

\usepackage{bm}
\newcommand{\bb}{\bm{\beta}}
\newcommand{\dif}{\mathrm{d}}
\newcommand{\bg}{\bm{\gamma}}
\newcommand{\bth}{\bm{\theta}}
\newcommand{\bx}{\bm{x}}
\newcommand{\ox}{\overline{x}}
\newcommand{\obx}{\overline{\bm{x}}}

%% maxwidth is the original width if it is less than linewidth
%% otherwise use linewidth (to make sure the graphics do not exceed the margin)
\makeatletter
\def\maxwidth{ %
  \ifdim\Gin@nat@width>\linewidth
    \linewidth
  \else
    \Gin@nat@width
  \fi
}
\makeatother

\definecolor{fgcolor}{rgb}{0.345, 0.345, 0.345}
\newcommand{\hlnum}[1]{\textcolor[rgb]{0.686,0.059,0.569}{#1}}%
\newcommand{\hlstr}[1]{\textcolor[rgb]{0.192,0.494,0.8}{#1}}%
\newcommand{\hlcom}[1]{\textcolor[rgb]{0.678,0.584,0.686}{\textit{#1}}}%
\newcommand{\hlopt}[1]{\textcolor[rgb]{0,0,0}{#1}}%
\newcommand{\hlstd}[1]{\textcolor[rgb]{0.345,0.345,0.345}{#1}}%
\newcommand{\hlkwa}[1]{\textcolor[rgb]{0.161,0.373,0.58}{\textbf{#1}}}%
\newcommand{\hlkwb}[1]{\textcolor[rgb]{0.69,0.353,0.396}{#1}}%
\newcommand{\hlkwc}[1]{\textcolor[rgb]{0.333,0.667,0.333}{#1}}%
\newcommand{\hlkwd}[1]{\textcolor[rgb]{0.737,0.353,0.396}{\textbf{#1}}}%
\let\hlipl\hlkwb

\usepackage{framed}
\makeatletter
\newenvironment{kframe}{%
 \def\at@end@of@kframe{}%
 \ifinner\ifhmode%
  \def\at@end@of@kframe{\end{minipage}}%
  \begin{minipage}{\columnwidth}%
 \fi\fi%
 \def\FrameCommand##1{\hskip\@totalleftmargin \hskip-\fboxsep
 \colorbox{shadecolor}{##1}\hskip-\fboxsep
     % There is no \\@totalrightmargin, so:
     \hskip-\linewidth \hskip-\@totalleftmargin \hskip\columnwidth}%
 \MakeFramed {\advance\hsize-\width
   \@totalleftmargin\z@ \linewidth\hsize
   \@setminipage}}%
 {\par\unskip\endMakeFramed%
 \at@end@of@kframe}
\makeatother

\definecolor{shadecolor}{rgb}{.97, .97, .97}
\definecolor{messagecolor}{rgb}{0, 0, 0}
\definecolor{warningcolor}{rgb}{1, 0, 1}
\definecolor{errorcolor}{rgb}{1, 0, 0}
\newenvironment{knitrout}{}{} % an empty environment to be redefined in TeX

% added for using R markdown
\usepackage{fancyvrb}
\newcommand{\VerbBar}{|}
\newcommand{\VERB}{\Verb[commandchars=\\\{\}]}
\DefineVerbatimEnvironment{Highlighting}{Verbatim}{commandchars=\\\{\}}
% Add ',fontsize=\small' for more characters per line
\usepackage{framed}
\definecolor{shadecolor}{RGB}{248,248,248}
\newenvironment{Shaded}{\begin{snugshade}}{\end{snugshade}}
\newcommand{\AlertTok}[1]{\textcolor[rgb]{0.94,0.16,0.16}{#1}}
\newcommand{\AnnotationTok}[1]{\textcolor[rgb]{0.56,0.35,0.01}{\textbf{\textit{#1}}}}
\newcommand{\AttributeTok}[1]{\textcolor[rgb]{0.77,0.63,0.00}{#1}}
\newcommand{\BaseNTok}[1]{\textcolor[rgb]{0.00,0.00,0.81}{#1}}
\newcommand{\BuiltInTok}[1]{#1}
\newcommand{\CharTok}[1]{\textcolor[rgb]{0.31,0.60,0.02}{#1}}
\newcommand{\CommentTok}[1]{\textcolor[rgb]{0.56,0.35,0.01}{\textit{#1}}}
\newcommand{\CommentVarTok}[1]{\textcolor[rgb]{0.56,0.35,0.01}{\textbf{\textit{#1}}}}
\newcommand{\ConstantTok}[1]{\textcolor[rgb]{0.00,0.00,0.00}{#1}}
\newcommand{\ControlFlowTok}[1]{\textcolor[rgb]{0.13,0.29,0.53}{\textbf{#1}}}
\newcommand{\DataTypeTok}[1]{\textcolor[rgb]{0.13,0.29,0.53}{#1}}
\newcommand{\DecValTok}[1]{\textcolor[rgb]{0.00,0.00,0.81}{#1}}
\newcommand{\DocumentationTok}[1]{\textcolor[rgb]{0.56,0.35,0.01}{\textbf{\textit{#1}}}}
\newcommand{\ErrorTok}[1]{\textcolor[rgb]{0.64,0.00,0.00}{\textbf{#1}}}
\newcommand{\ExtensionTok}[1]{#1}
\newcommand{\FloatTok}[1]{\textcolor[rgb]{0.00,0.00,0.81}{#1}}
\newcommand{\FunctionTok}[1]{\textcolor[rgb]{0.00,0.00,0.00}{#1}}
\newcommand{\ImportTok}[1]{#1}
\newcommand{\InformationTok}[1]{\textcolor[rgb]{0.56,0.35,0.01}{\textbf{\textit{#1}}}}
\newcommand{\KeywordTok}[1]{\textcolor[rgb]{0.13,0.29,0.53}{\textbf{#1}}}
\newcommand{\NormalTok}[1]{#1}
\newcommand{\OperatorTok}[1]{\textcolor[rgb]{0.81,0.36,0.00}{\textbf{#1}}}
\newcommand{\OtherTok}[1]{\textcolor[rgb]{0.56,0.35,0.01}{#1}}
\newcommand{\PreprocessorTok}[1]{\textcolor[rgb]{0.56,0.35,0.01}{\textit{#1}}}
\newcommand{\RegionMarkerTok}[1]{#1}
\newcommand{\SpecialCharTok}[1]{\textcolor[rgb]{0.00,0.00,0.00}{#1}}
\newcommand{\SpecialStringTok}[1]{\textcolor[rgb]{0.31,0.60,0.02}{#1}}
\newcommand{\StringTok}[1]{\textcolor[rgb]{0.31,0.60,0.02}{#1}}
\newcommand{\VariableTok}[1]{\textcolor[rgb]{0.00,0.00,0.00}{#1}}
\newcommand{\VerbatimStringTok}[1]{\textcolor[rgb]{0.31,0.60,0.02}{#1}}
\newcommand{\WarningTok}[1]{\textcolor[rgb]{0.56,0.35,0.01}{\textbf{\textit{#1}}}}

\usepackage{alltt}
\usepackage{graphicx}
\usepackage[margin=1cm]{geometry}
\usepackage{color}
\usepackage[pages=absolute]{flowfram}
\usepackage{lipsum}
\usepackage{url, pdfpages}
\usepackage{enumitem}
\setlist{itemsep = 0pt, topsep = 1pt, leftmargin = 0.6mm}
\usepackage[anythingbreaks]{breakurl}
\usepackage{microtype}
\usepackage{anyfontsize}
\usepackage{caption}
\usepackage{wrapfig}
\usepackage{scrextend}
\usepackage{booktabs}

% added
\usepackage{natbib}

\renewcommand{\UrlBreaks}{\do\/\do\a\do\b\do\c\do\d\do\e\do\f\do\g\do\h\do\i\do\j\do\k\do\l\do\m\do\n\do\o\do\p\do\q\do\r\do\s\do\t\do\u\do\v\do\w\do\x\do\y\do\z\do\A\do\B\do\C\do\D\do\E\do\F\do\G\do\H\do\I\do\J\do\K\do\L\do\M\do\N\do\O\do\P\do\Q\do\R\do\S\do\T\do\U\do\V\do\W\do\X\do\Y\do\Z}

\usepackage{algorithm, algorithmic}

%% for knitr
\usepackage[unicode=true,pdfusetitle,
 bookmarks=true,bookmarksnumbered=true,bookmarksopen=true,bookmarksopenlevel=2,
 breaklinks=false,pdfborder={0 0 1},backref=false,colorlinks=false]
 {hyperref}
\hypersetup{
 pdfstartview={XYZ null null 1}}

% graph path
\graphicspath{{images/}}
\DeclareGraphicsExtensions{.eps,. ps,. pdf, .jpg, .png}

\twocolumn

\begin{document}


% \section*{Software Review}
\section*{Recurrent Event Analysis with \pkg{reda} and \pkg{reReg}}

Recurrent event data arise in situations where the event of interest,
such as hospital admissions, infections, or tumor recurrences,
can recur in the same individual during follow-up.
The standard ``time-to-first'' event analysis cannot capture the cumulative experience of
the recurrent events and could lead to invalid inferences.
The \proglang{R} packages \pkg{reda} \citep{reda-package} and \pkg{reReg} \citep{reReg-package}
provide a collection of visualization tools and statistical methods for exploring and analyzing 
recurrent event data.


Suppose in a study consists of a random sample of $n$ subjects and 
$N_i(t)$ be the number of events the $i$th subject experienced over the interval $[0, t]$. 
Let $D$ be the failure time of interest that could either be a terminal event (e.g., death)
or a non-terminal event (e.g., treatment failure).
Let $C$ be the potential censoring time for reasons other than the failure event,
the observed data are independent and identically distributed copies
$\{N_i(t), Y_i, X_i; t\le Y_i, i = 1, \ldots, n\}$,
where $Y_i = \min(D, C)$, $\Delta_i = I(D\le C)$,
$X_i$ is a covariate vector, $I(\cdot)$ is the indicator function,
and the recurrent event process $N_i(\cdot)$ is observed up to the composite censoring time $Y_i$.
Suppose we are interested in making inference about the recurrent event process and the failure
event in the time interval $[0, \tau]$, where the constant $\tau$ is determined with the knowledge
that recurrent and failure events could potentially be observed up to time $\tau$.
We will illustrate the rehospitalization data \citep{gonzalez2005sex} from the \pkg{frailtypack} 
package \citep{fp}.

In the packages \pkg{reda} and \pkg{reReg}, recurrent event data are
represented using an object of type \texttt{Recur} created by the
\texttt{Recur()} function. The \texttt{Recur} object is an \texttt{S4}
class object that bundles together a set of recurrent times, failure
time, and censoring status, allowing users an easy first glance of the
recurrent event data. The \texttt{Recur} object is also used as the
formula response for many key functions in \pkg{reda} and \pkg{reReg}.
The following commands can be used to create a \texttt{Recur} object
corresponding to the rehospitalization data:

\begin{Shaded}
\begin{Highlighting}[]
\KeywordTok{library}\NormalTok{(reda); }\KeywordTok{library}\NormalTok{(reReg)}
\KeywordTok{data}\NormalTok{(readmission, }\DataTypeTok{package =} \StringTok{"frailtypack"}\NormalTok{)}
\KeywordTok{with}\NormalTok{(readmission, }\KeywordTok{Recur}\NormalTok{(t.stop, id, event, death))}
\end{Highlighting}
\end{Shaded}

\begin{verbatim}
Error: Subjects having multiple terminal events:
60, 109, 280.
\end{verbatim}

The \texttt{Recur()} internally checks whether the specified data fits
into the recurrent event data framework and detected a possible issue on
the data structure. The \texttt{show()} method for \texttt{Recur}
objects presents recurrent events in intervals, where events happened at
end of the recurrent episodes with censoring due to (or not) terminal
indicated by a trailing \texttt{+} (or \texttt{*}).

\begin{Shaded}
\begin{Highlighting}[]
\NormalTok{readmission0 \textless{}{-}}\StringTok{ }\KeywordTok{subset}\NormalTok{(readmission, id }\OperatorTok{\%in\%}\StringTok{ }\DecValTok{1}\OperatorTok{:}\DecValTok{5}\NormalTok{)}
\KeywordTok{with}\NormalTok{(readmission0, }\KeywordTok{Recur}\NormalTok{(t.stop, id, event, death))}
\end{Highlighting}
\end{Shaded}

\begin{verbatim}
[1] 1: (0, 24], (24, 457], (457, 1037+]       
[2] 2: (0, 489], (489, 1182+]                 
[3] 3: (0, 15], (15, 783*]                    
[4] 4: (0, 163], (163, 288], ..., (686, 2048+]
[5] 5: (0, 1134], (1134, 1144+]               
\end{verbatim}

Another quick and easy way to glance at recurrent event data is by event
plots, which can be created by directly applying the generic function
\texttt{plot()} to the \texttt{Recur} object when the \pkg{reReg}
package is loaded. Additionally, the \texttt{plotEvents()} function from
the \pkg{reReg} package allows users to stratify the event plots by
discrete variables. The following codes demonstrate these features.

\begin{Shaded}
\begin{Highlighting}[]
\NormalTok{df0 <-}\StringTok{ }\KeywordTok{subset}\NormalTok{(readmission, !(id %in%}\StringTok{ }\KeywordTok{c}\NormalTok{(}\DecValTok{60}\NormalTok{, }\DecValTok{109}\NormalTok{, }\DecValTok{280}\NormalTok{)))}
\NormalTok{obj <-}\StringTok{ }\KeywordTok{with}\NormalTok{(df0, }\KeywordTok{Recur}\NormalTok{(t.stop, id, event, death))}
\KeywordTok{plot}\NormalTok{(obj, }\DataTypeTok{legend =} \StringTok{"bottom"}\NormalTok{) ## no stratification}
\NormalTok{fn <-}\StringTok{ }\KeywordTok{Recur}\NormalTok{(t.stop, id, event, death) ~}\StringTok{ }\NormalTok{chemo}
\KeywordTok{plotEvents}\NormalTok{(fn, }\DataTypeTok{data =} \NormalTok{df0, }\DataTypeTok{legend =} \StringTok{"bottom"}\NormalTok{) ## by chemo}
\end{Highlighting}
\end{Shaded}

\vspace*{-.3cm}\begin{figure}[H]
\centering
\begin{subfigure}[t]{1.8in}
\centering
\includegraphics[scale = .125]{images/ep-1}
\caption{No stratification}\label{fig:1a}
\end{subfigure}
\quad
\begin{subfigure}[t]{1.8in}
\centering
\includegraphics[scale = .125]{images/ep-2}
\caption{Stratified by \texttt{chemo}}\label{fig:1b}
\end{subfigure}
\caption{Event plots}\label{fig:1}
\end{figure}

The \texttt{plot()} method, as well as the \texttt{plotEvents()}
function, return a \texttt{ggplot2} object \citep{hadley2016ggplot2} so
that users may further customize the plot easily.


\renewcommand{\bibsection}{\section*{Reference}}
\setlength{\bibhang}{0pt}
\setlength{\bibsep}{0.4em}
\bibliographystyle{asa}
\bibliography{ref}

\medskip

\noindent
\begin{minipage}[b]{1.2in}\centering
  \sbox0{\includegraphics{WenjieWang}}%
  \includegraphics[width=1.2in, trim={{0.08\wd0} {0.08\ht0} {0.08\wd0}
    {0.08\ht0}}, clip=true]{WenjieWang}\\
\end{minipage}
\hspace{0.2cm}
\begin{minipage}[b]{2.4in}
\begin{flushright}
  \emph{Wenjie Wang}\\
  Research Scientist\\
  Machine Learning, Artificial Intelligence, and Connected Care\\
  Advanced Analytics and Data Sciences\\
  Eli Lilly and Company\\
  Email: \email{wang\_wenjie@lilly.com}
\end{flushright}
\end{minipage}

\begin{minipage}[b]{1.2in}\centering
  \sbox0{\includegraphics{Chiou}}%
  \includegraphics[width=1.0in]{Chiou}\\
\end{minipage}
\hspace{0.2cm}
\begin{minipage}[b]{2.4in}
\begin{flushright}
  \emph{Sy Han Chiou}\\
  Assistant Professor\\
  Department of Mathematical Sciences\\
  University of Texas at Dallas\\
  Email: \email{schiou@utdallas.edu}
\end{flushright}
\end{minipage}


\end{document}
