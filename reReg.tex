The \texttt{reReg()} function from the \pkg{reReg} package provides
methods to fit semiparametric regression models to recurrent event data.
A general joint model for the rate function of the recurrent event
process and the hazard function of the failure time can be formulated as
follow:

\begin{equation}
\lambda(t) = Z \lambda_0(te^{X^\top\alpha})e^{X^\top\beta};
h(t) = Z h_0(te^{X^\top\eta})e^{X^\top\theta},\label{general}
\end{equation}

where \(Z\) is a latent shared frailty variable to account for
association between the two types of outcomes, \(\lambda_0(\cdot)\) is
the baseline rate function, \(h_0(\cdot)\) is the baseline hazard
function, and the regression coefficients \((\alpha, \eta)\) and
\((\beta, \theta)\) correspond to the shape and size parameters of the
rate function and hazard function, respectively. In contrast to many
shared-frailty models that require a parametric assumption, following
the idea of \citet{wang2001analyzing}, the \texttt{reReg()} function
implements semiparametric estimation procedures that do not require the
knowledge about the frailty distribution. As a result, the dependence
between recurrent events and failure event is left unspecified and the
proposed implementations accommodate informative censoring.

Model \eqref{general} includes several popular semiparametric models as
special cases, which can be specified via the \texttt{method} argument
with the rate function and hazard function separated by
``\texttt{\textbar{}}''. For examples, the joint Cox model of
\citet{huang2004joint} is a special case of \eqref{general} when
\(\alpha = \eta = 0\) and can be called by
\texttt{method\ =\ "cox\textbar{}cox"}; the joint accelerated mean model
of \citet{xu2017joint} is a special case when \(\alpha = \beta\) and
\(\eta = \theta\) and can be called by
\texttt{method\ =\ "am\textbar{}am"}. Treating the terminal event as
nuisances (\(\eta = \theta = 0\)), \eqref{general} reduces to the
generalized scale-change model of \citet{xu2019generalized}, called by
\texttt{method\ =\ "sc\textbar{}."}. Moreover, users can mix the models
depending on the application. For example,
\texttt{method\ =\ "cox\textbar{}ar"} postulate a Cox proportional model
for the recurrent event rate function and an accelerated rate model for
the terminal event hazard function (\(\alpha = \theta = 0\) in
\eqref{general}). For inference, the asymptotic variance is estimated
from an efficient resampling-based sandwich estimator motivated by
\citet{zeng2008efficient}. The resampling approach is faster than the
conventional bootstrap as it only requires evaluating perturbed
estimating equations rather than solving them. The following code fits
the joint Cox model with 200 (default) resampling replicates.

\begin{Shaded}
\begin{Highlighting}[]
\NormalTok{fit <-}\StringTok{ }\KeywordTok{reReg}\NormalTok{(fn, df0, }\DataTypeTok{method =} \StringTok{"cox|cox"}\NormalTok{); }\KeywordTok{summary}\NormalTok{(fit)}
\end{Highlighting}
\end{Shaded}

\begin{verbatim}
Call: 
reReg(formula = fn, data = df0, method = "cox|cox")

Recurrent event process:
             Estimate StdErr z.value p.value
chemoTreated   -0.189  0.243  -0.779   0.436

Terminal event:
             Estimate StdErr z.value p.value
chemoTreated    0.519  0.311   1.669   0.095
\end{verbatim}

The model shows that patients treated with chemotherapy are hospitalized
less often (17\% lower) and have a higher risk of death (68\% more) than
patients not treated with chemotherapy though not significant at the
0.05 level. After a model is fitted, the baseline rate function and
hazard function can be visualized by plotting the \texttt{reReg()}
object. See \url{wenjie-stat.me/reda/} and \url{www.sychiou.com/reReg/}
for the full package documents.
