When the study interest is placed on evaluating the covariate effect on
the recurrent event process and/or the terminal event, the
\texttt{reReg()} function from the \pkg{reReg} package provides
approaches to fit semiparametric regression model to recurrent event
data. The \texttt{reReg()} function offers users options to choose from
some of the most common models. In general, a joint scale-change model
for the rate function of the recurrent event process and the hazard
function of the failure time is formulated as follow:

\begin{equation}
\lambda(t) = Z \lambda_0(te^{X^\top\alpha})e^{X^\top\beta};
h(t) = Z h_0(te^{X^\top\eta})e^{X^\top\theta},
\label{general}
\end{equation}

where \(Z\) is a latent shared frailty variable to account for
association between the two types of outcomes, \(\lambda_0(\cdot)\) is
the baseline rate function, \(h_0(\cdot)\) is the baseline hazard
function, and the regression coefficients \((\alpha, \eta)\) and
\((\beta, \theta)\) correspond to the shape and size parameters\\
of the rate function and hazard function, respectively. In contrast to
usual shared-frailty models that require a parametric assumption on the
distribution of the frailty random variable, the \texttt{reReg()}
function provides semiparametric estimation procedures that do not
require the knowledge about the frailty as motivated by
\citet{wang2001analyzing}. As a result, the dependence between recurrent
events and failure event is left unspecified and the proposed
implementations accommodate informative censoring.

Model \ref{general} allows great flexibility and includes several
popular semiparametric models as special cases. For examples, the joint
Cox model proposed by \citet{huang2004joint} is a special case when
\(\alpha = \eta = 0\) and the joint Accelerated Mean model proposed by
\citet{xu2017joint} is a special case when \(\alpha = \eta\). Moreover,
when the primary interest is in the covariate effects on the risk of
recurrent events and treating the terminal event as nuisances, i.e.,
\(\eta = \theta = 0\), model \ref{general} reduces to the generalized
scale-change model proposed in \citet{xu2019generalized}. The different
model types can be specified via the \texttt{method} argument with the
type of rate function and hazard separated by \texttt{\textbar{}}. For
example, setting \texttt{method\ =\ "cox\textbar{}cox"} gives the joint
Cox model of \citet{huang2004joint} and
\texttt{method\ =\ "sc\textbar{}."} gives the generalized scale-change
model of \citet{xu2019generalized}. Other options includes \texttt{ar}
for the accerlated rate models (\(\beta = \theta = 0\)) and \texttt{am}
for the accerlated mean models. The following codes fit the joint Cox
model and the generalized scale-change model respectively.

\begin{Shaded}
\begin{Highlighting}[]
\NormalTok{fit1 <-}\StringTok{ }\KeywordTok{reReg}\NormalTok{(fn, }\DataTypeTok{data =} \NormalTok{df0, }\DataTypeTok{method =} \StringTok{"cox|cox"}\NormalTok{, }\DataTypeTok{se =} \StringTok{"r"}\NormalTok{)}
\NormalTok{fit2 <-}\StringTok{ }\KeywordTok{reReg}\NormalTok{(fn, }\DataTypeTok{data =} \NormalTok{df0, }\DataTypeTok{method =} \StringTok{"sc|."}\NormalTok{, }\DataTypeTok{se =} \StringTok{"r"}\NormalTok{)}
\end{Highlighting}
\end{Shaded}
