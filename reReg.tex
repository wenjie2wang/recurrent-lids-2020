When the study interest is to evaluate the covariate effect on the
recurrent event process and/or the terminal event, the \texttt{reReg()}
function from the \pkg{reReg} package fits semiparametric regression
models to recurrent event data. A general joint model for the rate of
the recurrent event process and the hazard of the failure time can be
formulated as follow:

\begin{equation}
\lambda(t) = Z \lambda_0(te^{X^\top\alpha})e^{X^\top\beta};
h(t) = Z h_0(te^{X^\top\eta})e^{X^\top\theta},
\label{general}
\end{equation}

where \(Z\) is a latent shared frailty variable to account for
association between the two types of outcomes, \(\lambda_0(\cdot)\) is
the baseline rate function, \(h_0(\cdot)\) is the baseline hazard
function, and the regression coefficients \((\alpha, \eta)\) and
\((\beta, \theta)\) correspond to the shape and size parameters of the
rate function and hazard function, respectively. In contrast to many
shared-frailty models that require a parametric assumption, following
the idea of \citet{wang2001analyzing}, the \texttt{reReg()} function
implements semiparametric estimation procedures that do not require the
knowledge about the frailty distribution. As a result, the dependence
between recurrent events and failure event is left unspecified and the
proposed implementations accommodate informative censoring.

Model \eqref{general} includes several popular semiparametric models as
special cases, which can be specified via the \texttt{method} argument
with the rate function and hazard function separated by
``\texttt{\textbar{}}''. For examples, the joint Cox model proposed by
\citet{huang2004joint} is a special case of \eqref{general} when
\(\alpha = \eta = 0\) and can be called by
\texttt{method\ =\ "cox\textbar{}cox"}; the joint Accelerated Mean model
proposed by \citet{xu2017joint} is a special case when
\(\alpha = \beta\) and \(\eta = \theta\) and can be called by
\texttt{method\ =\ "am\textbar{}am"}. When the primary interest is in
the covariate effects on the risk of recurrent events and treating the
terminal event as nuisances, i.e., \(\eta = \theta = 0\), model
\eqref{general} reduces to the generalized scale-change model proposed
in \citet{xu2019generalized}, called by
\texttt{method\ =\ "sc\textbar{}."}. Users can mixed and match the
\texttt{method} arguments depending on the application. For example,
\texttt{method\ =\ "cox\textbar{}ar"} postulate a Cox proportional model
for the recurrent event rate function and an accelerated rate model for
the terminal event hazard function, i.e., this assumes
\(\alpha = \theta = 0\) in \eqref{general}. For inference, the
asymptotic variance is estimated from an efficient resampling-based
sandwich estimator motivated by \citet{zeng2008efficient}. The
resampling approach is faster than the conventional bootstrap method as
it only requires evaluating perturbed estimating equations rather than
solving them. The following code fit the joint Cox model with 200
(default) resampling replicates:

\begin{Shaded}
\begin{Highlighting}[]
\KeywordTok{system.time}\NormalTok{(fit <-}\StringTok{ }\KeywordTok{reReg}\NormalTok{(fn, df0, }\DataTypeTok{method =} \StringTok{"cox|cox"}\NormalTok{))}
\end{Highlighting}
\end{Shaded}

\begin{verbatim}
   user  system elapsed 
  1.884   0.016   1.903 
\end{verbatim}

The \texttt{summary()} method prints the results of the model fits:

\begin{Shaded}
\begin{Highlighting}[]
\KeywordTok{summary}\NormalTok{(fit)}
\end{Highlighting}
\end{Shaded}

\begin{verbatim}
Call: 
reReg(formula = fn, data = df0, method = "cox|cox")

Recurrent event process:
             Estimate StdErr z.value p.value
chemoTreated   -0.189  0.244  -0.778   0.437

Terminal event:
             Estimate StdErr z.value p.value  
chemoTreated    0.519  0.286   1.815    0.07 .
\end{verbatim}

After a model is fitted, the baseline rate function and hazard function
can be visualized by plotting the \texttt{reReg()} object. Methods
proposed by authors \citet{lin2000semiparametric},
\citet{ghosh2002marginal}, and \citet{ghosh2003semiparametric} are also
implemented in \pkg{reReg}. See \url{wenjie-stat.me/reda/} and
\url{www.sychiou.com/reReg/} for the full package documents.
