In the packages \pkg{reda} and \pkg{reReg}, recurrent event data are
represented using an object of type \texttt{Recur} created by the
\texttt{Recur()} function. The \texttt{Recur} object is an \texttt{S4}
class object that bundles together a set of recurrent times, failure
time, and censoring status, allowing users an easy first glance of the
recurrent event data. The \texttt{Recur} object is also used as the
formula response for many key functions in \pkg{reda} and \pkg{reReg}.
The following commands can be used to create a \texttt{Recur} object
corresponding to the rehospitalization data:

\begin{Shaded}
\begin{Highlighting}[]
\KeywordTok{library}\NormalTok{(reda); }\KeywordTok{library}\NormalTok{(reReg)   }
\KeywordTok{data}\NormalTok{(readmission, }\DataTypeTok{package =} \StringTok{"frailtypack"}\NormalTok{)}
\KeywordTok{with}\NormalTok{(readmission, }\KeywordTok{Recur}\NormalTok{(t.stop, id, event, death))}
\end{Highlighting}
\end{Shaded}

\begin{verbatim}
## Error: Subjects having multiple terminal events:
## 60, 109, 280.
\end{verbatim}

The \texttt{Recur()} internally calls \texttt{check\_Recur()} to check
whether the specified data fits into the recurrent event data framework
and detected a possible issue on the data structure. Without these
subjects, the \texttt{Recur} object is presented by intervals

\begin{Shaded}
\begin{Highlighting}[]
\NormalTok{readmission0 <-}
\StringTok{    }\KeywordTok{subset}\NormalTok{(readmission, !(id %in%}\StringTok{ }\KeywordTok{c}\NormalTok{(}\DecValTok{60}\NormalTok{, }\DecValTok{109}\NormalTok{, }\DecValTok{280}\NormalTok{)))}
\KeywordTok{options}\NormalTok{(}\DataTypeTok{max.print =} \DecValTok{5}\NormalTok{)}
\NormalTok{(obj <-}\StringTok{ }\KeywordTok{with}\NormalTok{(readmission0,}
            \KeywordTok{Recur}\NormalTok{(t.stop, id, event, death)))}
\end{Highlighting}
\end{Shaded}

\begin{verbatim}
## [1] 1: (0, 24], (24, 457], (457, 1037+]       
## [2] 2: (0, 489], (489, 1182+]                 
## [3] 3: (0, 15], (15, 783*]                    
## [4] 4: (0, 163], (163, 288], ..., (686, 2048+]
## [5] 5: (0, 1134], (1134, 1144+]               
##  [ reached getOption("max.print") -- omitted 395 entries ]
\end{verbatim}

For a concise printing, up to three recurrent intervals are printed.
Users can modify this by specifying \texttt{reda.Recur.maxPrint} in
\texttt{options()}.
