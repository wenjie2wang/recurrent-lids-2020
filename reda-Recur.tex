In the packages \pkg{reda} and \pkg{reReg}, recurrent event data are
represented using an object of type \texttt{Recur} created by the
\texttt{Recur()} function. The \texttt{Recur} object is an \texttt{S4}
class object that bundles together a set of recurrent times, failure
time, and censoring status, allowing users an easy first glance of the
recurrent event data. The \texttt{Recur} object is also used as the
formula response for many key functions in \pkg{reda} and \pkg{reReg}.
The following commands can be used to create a \texttt{Recur} object
corresponding to the rehospitalization data:

\begin{Shaded}
\begin{Highlighting}[]
\KeywordTok{library}\NormalTok{(reda); }\KeywordTok{library}\NormalTok{(reReg)}
\KeywordTok{data}\NormalTok{(readmission, }\DataTypeTok{package =} \StringTok{"frailtypack"}\NormalTok{)}
\KeywordTok{with}\NormalTok{(readmission, }\KeywordTok{Recur}\NormalTok{(t.stop, id, event, death))}
\end{Highlighting}
\end{Shaded}

\begin{verbatim}
Error: Subjects having multiple terminal events:
60, 109, 280.
\end{verbatim}

The \texttt{Recur()} internally checks whether the specified data fits
into the recurrent event data framework and detected a possible issue on
the data structure. The \texttt{show()} method for \texttt{Recur}
objects presents recurrent events in intervals, where events happened at
end of the recurrent episodes with censoring due to (or not) terminal
indicated by a trailing \texttt{+} (or \texttt{*}). The following prints
the \texttt{Recur} object for the first five subjects.

\begin{Shaded}
\begin{Highlighting}[]
\KeywordTok{with}\NormalTok{(readmission[}\DecValTok{1}\NormalTok{:}\DecValTok{14}\NormalTok{,], }\KeywordTok{Recur}\NormalTok{(t.stop, id, event, death))}
\end{Highlighting}
\end{Shaded}

\begin{verbatim}
[1] 1: (0, 24], (24, 457], (457, 1037+]       
[2] 2: (0, 489], (489, 1182+]                 
[3] 3: (0, 15], (15, 783*]                    
[4] 4: (0, 163], (163, 288], ..., (686, 2048+]
[5] 5: (0, 1134], (1134, 1144+]               
\end{verbatim}
